% options:
% thesis=B bachelor's thesis
% thesis=M master's thesis
% czech thesis in Czech language
% slovak thesis in Slovak language
% english thesis in English language
% hidelinks remove colour boxes around hyperlinks

\documentclass[thesis=B,czech]{FITthesis}[2012/06/26]

\usepackage[utf8]{inputenc}

% \usepackage[unicode]{hyperref}

\usepackage{graphicx} %graphics files inclusion
% \usepackage{amsmath} %advanced maths
% \usepackage{amssymb} %additional math symbols

\usepackage{dirtree} %directory tree visualisation

% % list of acronyms
% \usepackage[acronym,nonumberlist,toc,numberedsection=autolabel]{glossaries}
% \iflanguage{czech}{\renewcommand*{\acronymname}{Seznam pou{\v z}it{\' y}ch zkratek}}{}
% \makeglossaries

\newcommand{\tg}{\mathop{\mathrm{tg}}} %cesky tangens
\newcommand{\cotg}{\mathop{\mathrm{cotg}}} %cesky cotangens

% % % % % % % % % % % % % % % % % % % % % % % % % % % % % % 
% ODTUD DAL VSE ZMENTE
% % % % % % % % % % % % % % % % % % % % % % % % % % % % % % 

\department{Katedra \ldots (softwarového inženýrství)}
\title{ InfoWeb - Nástroj získávání informací z webů }
\authorGN{Jakub} %(křestní) jméno (jména) autora
\authorFN{Tuček} %příjmení autora
\authorWithDegrees{} %jméno autora včetně současných akademických titulů
\supervisor{Ing. Jiří Hunka}
\acknowledgements{Chtěl bych poděkovat za trpělivost vedoucímu, Ing. Jiřímu Hunkovi.}
\abstractCS{V~několika větách shrňte obsah a přínos této práce v~češtině. Po přečtení abstraktu
by se čtenář měl mít čtenář dost informací pro rozhodnutí, zda chce Vaši práci číst.}
\abstractEN{Sem doplňte ekvivalent abstraktu Vaší práce v~angličtině.}
\placeForDeclarationOfAuthenticity{V~Praze}
\declarationOfAuthenticityOption{4} %volba Prohlášení (číslo 1-6)
\keywordsCS{Nahraďte seznamem klíčových slov v češtině oddělených čárkou.}
\keywordsEN{Nahraďte seznamem klíčových slov v angličtině oddělených čárkou.}

\begin{document}

% \newacronym{CVUT}{{\v C}VUT}{{\v C}esk{\' e} vysok{\' e} u{\v c}en{\' i} technick{\' e} v Praze}
% \newacronym{FIT}{FIT}{Fakulta informa{\v c}n{\' i}ch technologi{\' i}}

\begin{introduction}

V předmětech BI-SP1 a BI-SP2 v prostředí FIT ČVUT byl realizován týmový projekt pro získávání informací z webů s primárním zaměřením na potřeby
obchodů. Projekt řešil problém automatizace získávání dat z webů, jelikož stávající služby neposkytují veřejné rozhraní
nebo mají velkou chybovost dat.
\par
Požadavky internetových obchodů jsou především tvořeny nutností držet krok s trhem a tedy sledovat vývoj cen
prodávaných produktů u konkurence. Teprve na základě těchto dat je možné reagovat na trh a měnit vlastní cenu produktů.
\par
Cílem této práce je popsat požadavky internetových obchodů, stávající stav a možná řešení. Dále na základě těchto poznatků
zhodnotit vytvořené řešení a včetně korektnosti zvolených postupů navrhnout vylepšení. Ty implementovat, řádně 
otestovat vylepšení a zhodnotit výsledný stav projektu.


\newpage

\newpage

\end{introduction}

\chapter{Cíl práce}

\section{Analytické cíle}

\begin{enumerate}  
\item Rešerše aktuálního stavu získávání dat pro potřeby
internetových obchodů
\item Analýza vzniklého řešení týmového projektu vzniklého v prostředí ČVUT FIT,
včetně důrazu na použité postupy při softwarovém vývoji
\item Návrh a zhodnocení implementovaných vylepšení
\end{enumerate}

\section{Praktické cíle}
\begin{enumerate}  
\item Implementace vylepšení systému
\end{enumerate}

\newpage


\chapter{Analýza a návrh}

V této kapitole se budu nejprve zabývat samotnou problematikou získávání informací 
z webů s důrazem na internetové obchody.
Jelikož je tato problematika již řešena existujícími službami, je nutné je zhodnotit a popsat
jejích chování.
Dále zhodnotím současný stav projektu, zvolené postupy při vývoji a výslednou funkcionalitu.
Nakonec navrhnu vylepšení vzniklého systému, ty nejdůležitější implementuji a výsledný stav zhodnotím.

\section{Získávání informací z webů}

\subsection{Problematika}
Získávání informací z webů je efektivní možnost jak získat databázi informací, které se na internetu vyskytují.
Tato činnost však stojí na problematice data získávat a uchovávat v potřebné struktuře, jelikož 
jinak z dat nejsme schopni vyčíst potřebné informace.
Vzhledem k specificitě dat, které jsou v kontextu činnosti zajímavá a dále kvůli unikátnosti webových stránek
není možné jednoznačně určit jak data získat v požadovaném formátu.
\subsection{Výběr dat}
Nejčastější řešení a jediné řešení je kombinace lidské inteligence a automatizování co nejvíce činností.
To je obvykle dosaženo roboty, která stahují data a lidské práce určující jaké informace nás zajímají.
\par
Získávání dat ze stažených stránek lze poté zjednodušit na problematiku lokace elementů v HTML, které jsou pro 
nás zajímavé.
Lokaci elementu v HTML se kterým je potřeba pracovat lze poté jednoznačně určit pomocí dvou možnosti:
\begin{enumerate}
\item XPath
\item CSS Selector
\end{enumerate}

\subsection{XML Path Language}
XML Path Language\cite{XPath} nazývaný zkráceně XPath je jazyk, který slouží k výběru elementu v dokumentu ve formátu XML.
\par
XML chápeme jako jazyk popisující strukturu dat, které jsou strojově i lidsky čitelné.
HTML je speciální typ XML, který popisuje obsah dat pro prezentaci ve webovém prohlížeči pomocí předem definované struktury,
které prohlížeče rozumí.\cite{XML} Díky této vlastnosti lze tedy použít XPath pro definování cesty k prvku (a jeho obsahu),
který uchovává potřebnou informaci na webové stránce.
\par
\subsection{CSS Selector}
Jazyk CSS je používán pro vizuální popis prezentace webové stránky definované v HTML. Jazyk k určení prvků se kterými
chce pracovat používá selektory, které označují prvek v HTML. Jako selektor může být použit jak samotný název prvku,
tak vlastní definované třídy.\cite{CSS}
\par
Pomocí řetězení těchto selektorů jsme schopni jednoznačně získat element v HTML.


\section{Současný stav řešení potřeb internetových obchodů}
I v kontextu malého trhu jako Česká republika se lze bavit o velké konkurenci na poli 
maloobchodů prodávající své zboží na internetu.
Internetové obchody potřebují monitorovat konkurenci a trh. Vzhledem k jejich zaměření je tedy nejvíce zajímají 
obchody prodávající stejné zboží. Potřebné informace o prodávaných produktů konkurencí 
se skládají z následujících hlavních atributů:

\begin{enumerate}
\item Cena
\item Inzerovaný název
\item Dostupnost
\end{enumerate}

\subsection{Srovnávače cen}

Data lze získat pomocí srovnávačů cen jako \textit{zbozi.cz}\cite{heureka} 
nebo \textit{heureka.cz}\cite{zbozi}. Problém u těchto služeb je však že jsou spíše určeny koncovým zákazníkům 
pro nalezení nejlepší ceny na trhu pro určitý produkt. S tím souvisí to, že největší srovnávače cen neposkytují data nebo
rozhraní přes která by je bylo možné získat.

\subsection{Existující služby}

Problematiku sledování trhu s důrazem na firemní klientelu, řeší aktuálně několik existujících služeb.
\par
Služby mají v zásadě velmi podobnou povahu služeb. Rámcově se jedná o porovnávání cen včetně historie na různých internetových
obchodem či na srovnávačích cen. Uživatel si zadá okruh či seznam produktů, buďto formou manuální či vstupem ze souboru, případně 
přímých napojením na e-shop. Následně je možné konkrétní data zobrazit v grafech označující vývoj cen, trendů či náhlých změn.
Dále umožňují externí výstup do souboru v dostupných formátech.
\par
Největší rozdíl služeb je zda jsou data získávána přímo z obchodů nebo ze srovnávačích. Další odlišností je 
možnost zda služba dokáže sledovat i zahraniční trh.
\par
Cena služeb se nejvíce odvíjí od počtu sledovaných produktů a četnosti aktualizací. Proto se měsíční platby mohou 
pohybovat od stovek korun po desítek tisíc korun.

\subsubsection{Price checking}

\subsubsection{Pricing intelligence}

\subsubsection{Sledování trhu}

\subsubsection{Pricebot}

\subsubsection{Zahraniční nástroje}
Tyto nástroje jsou obecněji zaměřené a obvykle požadují od uživatele techničtější zaměření, 
jelikož je nutné přesně specifikovat kde, co a jak chce sledovat. Vzhledem k tomuto omezení
nejsou přímo pro provozovatele e-shopů vhodné kvůli nedostatečným technickým kapacitám a 
pro tuto práci důležité.
\par
Bodový seznam zahraničních nástrojů:

\begin{enumerate}
\item Screen scraper \cite{Screen scraper}

  \begin{itemize}
    \item Webová služba
    \item procházení web skrz odkazy
    \item potvrzování formulářů
    \item využití interního vyhledávání
    \item export do širokého množství formátu souborů
    \item cena: \$549 - \$2,799 za měsíc
  \end{itemize}
  
\item Web extractor \cite{Web extractor}

  \begin{itemize}
    \item Windows Aplikace
    \item procházení zadaných stránek
    \item hledání stránek pomocí klíčových slov
    \item export do csv formátu
    \item cena: \$99 - \$199 jednorázově
  \end{itemize}

\item Web Scraper \cite{Web scraper}

\end{enumerate}

\section{Zhodnocení současného stavu projektu}
TODO
\section{Návrh na vylepšení}
TODO
\section{Analýza nového řešení}
TODO


\chapter{Realizace}

\section{Způsob realizace stávajícího řešení}
TODO
\section{Implementace vylepšení}
TODO

\begin{conclusion}
	%sem napište závěr Vaší práce
\end{conclusion}

\bibliographystyle{csn690}
\bibliography{mybibliographyfile}

\begin{thebibliography}{}

\bibitem{XPath}
	{XPath. In: Wikipedia: the free encyclopedia [online]. 
	San Francisco (CA): Wikimedia Foundation, 2001- [cit. 2017-04-01]. 
	Dostupné z: 
	\url{https://en.wikipedia.org/wiki/XPath}}
\bibitem{XML}
	{PERUGINI, Saverio. HTML versus XML. 
	In: Virginia Tech - College of engineering: Department of computer science [online]. 
	Virginia Tech: Virginia Tech, 2002 [cit. 2017-04-01]. Dostupné z: 
	\url{http://courses.cs.vt.edu/~cs1204/XML/htmlVxml.html}}
\bibitem{CSS}
	{Cascading Style Sheets. In: Wikipedia: the free encyclopedia [online]. 
	San Francisco (CA): Wikimedia Foundation, 2001- [cit. 2017-04-01]. 
	Dostupné z: 
	\url{https://en.wikipedia.org/wiki/Cascading_Style_Sheets}}	
\bibitem{heureka}
	{Heuréka [online]. [cit. 2017-04-01]. Dostupné z:
	\url{http://www.heureka.cz/}}
\bibitem{zbozi}
	{Zboží [online]. [cit. 2017-04-01]. Dostupné z:
	\url{http://www.zbozi.cz/}}	
	
	
	
	
\bibitem{Screen scraper}
	{Screen scraper [online]. [cit. 2017-04-01]. Dostupné z:
	\url{http://www.screen-scraper.com/}}	
\bibitem{Web extractor}
	{Web extractor [online]. [cit. 2017-04-01]. Dostupné z:
	\url{http://www.webextractor.com/}}		
\bibitem{Web scraper}
	{Web Scraper [online]. [cit. 2017-04-01]. Dostupné z:
	\url{http://http://webscraper.io/}}		
\appendix
\chapter{Seznam použitých zkratek}
% \printglossaries
\begin{description}
	\item[XML] Extensible markup language
	\item[HTML] Hypertext Markup Language
	\item[CSS] Cascading style sheets
\end{description}

\end{thebibliography}

% % % % % % % % % % % % % % % % % % % % % % % % % % % % 
% % Tuto kapitolu z výsledné práce ODSTRAŇTE.
% % % % % % % % % % % % % % % % % % % % % % % % % % % % 
% 
% \chapter{Návod k~použití této šablony}
% 
% Tento dokument slouží jako základ pro napsání závěrečné práce na Fakultě informačních technologií ČVUT v~Praze.
% 
% \section{Výběr základu}
% 
% Vyberte si šablonu podle druhu práce (bakalářská, diplomová), jazyka (čeština, angličtina) a kódování (ASCII, \mbox{UTF-8}, \mbox{ISO-8859-2} neboli latin2 a nebo \mbox{Windows-1250}). 
% 
% V~české variantě naleznete šablony v~souborech pojmenovaných ve formátu práce\_kódování.tex. Typ může být:
% \begin{description}
% 	\item[BP] bakalářská práce,
% 	\item[DP] diplomová (magisterská) práce.
% \end{description}
% Kódování, ve kterém chcete psát, může být:
% \begin{description}
% 	\item[UTF-8] kódování Unicode,
% 	\item[ISO-8859-2] latin2,
% 	\item[Windows-1250] znaková sada 1250 Windows.
% \end{description}
% V~případě nejistoty ohledně kódování doporučujeme následující postup:
% \begin{enumerate}
% 	\item Otevřete šablony pro kódování UTF-8 v~editoru prostého textu, který chcete pro psaní práce použít -- pokud můžete texty s~diakritikou normálně přečíst, použijte tuto šablonu.
% 	\item V~opačném případě postupujte dále podle toho, jaký operační systém používáte:
% 	\begin{itemize}
% 		\item v~případě Windows použijte šablonu pro kódování \mbox{Windows-1250},
% 		\item jinak zkuste použít šablonu pro kódování \mbox{ISO-8859-2}.
% 	\end{itemize}
% \end{enumerate}
% 
% 
% V~anglické variantě jsou šablony pojmenované podle typu práce, možnosti jsou:
% \begin{description}
% 	\item[bachelors] bakalářská práce,
% 	\item[masters] diplomová (magisterská) práce.
% \end{description}
% 
% \section{Použití šablony}
% 
% Šablona je určena pro zpracování systémem \LaTeXe{}. Text je možné psát v~textovém editoru jako prostý text, lze však také využít specializovaný editor pro \LaTeX{}, např. Kile.
% 
% Pro získání tisknutelného výstupu z~takto vytvořeného souboru použijte příkaz \verb|pdflatex|, kterému předáte cestu k~souboru jako parametr. Vhodný editor pro \LaTeX{} toto udělá za Vás. \verb|pdfcslatex| ani \verb|cslatex| \emph{nebudou} s~těmito šablonami fungovat.
% 
% Více informací o~použití systému \LaTeX{} najdete např. v~\cite{wikilatex}.
% 
% \subsection{Typografie}
% 
% Při psaní dodržujte typografické konvence zvoleného jazyka. České \uv{uvozovky} zapisujte použitím příkazu \verb|\uv|, kterému v~parametru předáte text, jenž má být v~uvozovkách. Anglické otevírací uvozovky se v~\LaTeX{}u zadávají jako dva zpětné apostrofy, uzavírací uvozovky jako dva apostrofy. Často chybně uváděný symbol "{} (palce) nemá s~uvozovkami nic společného.
% 
% Dále je třeba zabránit zalomení řádky mezi některými slovy, v~češtině např. za jednopísmennými předložkami a spojkami (vyjma \uv{a}). To docílíte vložením pružné nezalomitelné mezery -- znakem \texttt{\textasciitilde}. V~tomto případě to není třeba dělat ručně, lze použít program \verb|vlna|.
% 
% Více o~typografii viz \cite{kobltypo}.
% 
% \subsection{Obrázky}
% 
% Pro umožnění vkládání obrázků je vhodné použít balíček \verb|graphicx|, samotné vložení se provede příkazem \verb|\includegraphics|. Takto je možné vkládat obrázky ve formátu PDF, PNG a JPEG jestliže používáte pdf\LaTeX{} nebo ve formátu EPS jestliže používáte \LaTeX{}. Doporučujeme preferovat vektorové obrázky před rastrovými (vyjma fotografií).
% 
% \subsubsection{Získání vhodného formátu}
% 
% Pro získání vektorových formátů PDF nebo EPS z~jiných lze použít některý z~vektorových grafických editorů. Pro převod rastrového obrázku na vektorový lze použít rasterizaci, kterou mnohé editory zvládají (např. Inkscape). Pro konverze lze použít též nástroje pro dávkové zpracování běžně dodávané s~\LaTeX{}em, např. \verb|epstopdf|.
% 
% \subsubsection{Plovoucí prostředí}
% 
% Příkazem \verb|\includegraphics| lze obrázky vkládat přímo, doporučujeme však použít plovoucí prostředí, konkrétně \verb|figure|. Například obrázek \ref{fig:float} byl vložen tímto způsobem. Vůbec přitom nevadí, když je obrázek umístěn jinde, než bylo původně zamýšleno -- je tomu tak hlavně kvůli dodržení typografických konvencí. Namísto vynucování konkrétní pozice obrázku doporučujeme používat odkazování z~textu (dvojice příkazů \verb|\label| a \verb|\ref|).
% 
% \begin{figure}\centering
% 	\includegraphics[width=0.5\textwidth, angle=30]{cvut-logo-bw}
% 	\caption[Příklad obrázku]{Ukázkový obrázek v~plovoucím prostředí}\label{fig:float}
% \end{figure}
% 
% \subsubsection{Verze obrázků}
% 
% % Gnuplot BW i barevně
% Může se hodit mít více verzí stejného obrázku, např. pro barevný či černobílý tisk a nebo pro prezentaci. S~pomocí některých nástrojů na generování grafiky je to snadné.
% 
% Máte-li například graf vytvořený v programu Gnuplot, můžete jeho černobílou variantu (viz obr. \ref{fig:gnuplot-bw}) vytvořit parametrem \verb|monochrome dashed| příkazu \verb|set term|. Barevnou variantu (viz obr. \ref{fig:gnuplot-col}) vhodnou na prezentace lze vytvořit parametrem \verb|colour solid|.
% 
% \begin{figure}\centering
% 	\includegraphics{gnuplot-bw}
% 	\caption{Černobílá varianta obrázku generovaného programem Gnuplot}\label{fig:gnuplot-bw}
% \end{figure}
% 
% \begin{figure}\centering
% 	\includegraphics{gnuplot-col}
% 	\caption{Barevná varianta obrázku generovaného programem Gnuplot}\label{fig:gnuplot-col}
% \end{figure}
% 
% 
% \subsection{Tabulky}
% 
% Tabulky lze zadávat různě, např. v~prostředí \verb|tabular|, avšak pro jejich vkládání platí to samé, co pro obrázky -- použijte plovoucí prostředí, v~tomto případě \verb|table|. Například tabulka \ref{tab:matematika} byla vložena tímto způsobem.
% 
% \begin{table}\centering
% 	\caption[Příklad tabulky]{Zadávání matematiky}\label{tab:matematika}
% 	\begin{tabular}{|l|l|c|c|}\hline
% 		Typ		& Prostředí		& \LaTeX{}ovská zkratka	& \TeX{}ovská zkratka	\tabularnewline \hline \hline
% 		Text		& \verb|math|		& \verb|\(...\)|	& \verb|$...$|		\tabularnewline \hline
% 		Displayed	& \verb|displaymath|	& \verb|\[...\]|	& \verb|$$...$$|	\tabularnewline \hline
% 	\end{tabular}
% \end{table}
% 
% % % % % % % % % % % % % % % % % % % % % % % % % % % % 

\chapter{Obsah přiloženého CD}

%upravte podle skutecnosti

\begin{figure}
	\dirtree{%
		.1 readme.txt\DTcomment{stručný popis obsahu CD}.
		.1 exe\DTcomment{adresář se spustitelnou formou implementace}.
		.1 src.
		.2 impl\DTcomment{zdrojové kódy implementace}.
		.2 thesis\DTcomment{zdrojová forma práce ve formátu \LaTeX{}}.
		.1 text\DTcomment{text práce}.
		.2 thesis.pdf\DTcomment{text práce ve formátu PDF}.
		.2 thesis.ps\DTcomment{text práce ve formátu PS}.
	}
\end{figure}

\end{document}
